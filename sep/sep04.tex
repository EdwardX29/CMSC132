\section{Wednesday, September 4, 2019}

\subsection{The Comparable Interface}

The \vocab{comparable interface} is used in order to provide an ordering to objects of a user-defined class. When a class implements this interface, it must implement the \verb!.compareTo()! method, which returns a negative value if the current object is less than the object being compared to, zero if they're equal, and positive otherwise.

As an example, consider the following code segment:

\begin{lstlisting}
package equalsComparable;
import java.util.ArrayList;

public class Example {

    public static void main(String[] args) {
        ArrayList<String> l = new ArrayList<String>();
        l.add("ice_cream");
        l.add("apple");
        l.add("raw_onion");
    }
}
\end{lstlisting}


If we were to print this ArrayList with \verb!System.out.println(l)!, we'd get the output \verb![ice_cream, apple, raw_onion]!, which is the order in which we inserted the elements into the ArrayList.


But now, what if we wanted to sort our elements and then print them? Calling \verb!Collections.sort(l)! and then printing would sort our items lexicographically, so our output would be \verb![apple, ice_cream, raw_onion]!. Why does this happen? Because the string class has a built-in \verb!compareTo! method, which sorts lexicographically. 


It turns out that we can define our own \verb!compareTo()! method and sort our elements according to how we want. As mentioned previously, this can be done by implementing the \verb!Comparable! interface. 


Consider the following  code segment, which illustrates how we can use the \verb!Comparable! interface:



\begin{lstlisting}
public class Student implements Comparable<Student> {
	private String name;
	private int id;

	public Student(String name, int id) {
		this.name = name;
		this.id = id;
	}

	public String toString() {
		return "Name: " + name + ", Id: " + id;
	}

	public int compareTo(Student other) {
		if (id < other.id) {
			return -1;
		} else if (id == other.id) {
			return 0;
		} else {
			return 1;
		}
	}

	public boolean equals(Object obj) {
		if (obj == this) {
			return true;
		} else if (!(obj instanceof Student)) {
			return false;
		} else {
			return compareTo((Student) obj) == 0;
		}
	}

	/* If we override equals we must have correct hashCode */
	public int hashCode() {
		return id;
	}
}
\end{lstlisting}

Note that the \verb!Student! class implements the \verb!Comparable! interface (whatever comes in the angled brackets \verb!< ... >! specifies what we will be comparing). Consequently, the  \verb!compareTo! method is overriden so that we can specify that we want to sort the students based on their IDs (we retunr a negative value when the student's ID is less than the other students, zero when they're equal, and a positive value otherwise). Now, if we use \verb!Collections.sort()! on an ArrayList of \verb!Student! objects, we will sort based on ID. 

An even simpler implementation of \verb!compareTo! that would do just what we want would be to only have the statement \verb!return id - other.id!.

Finally, we have a \verb!hashCode! method, which will be discussed in-depth at a later time. But for now, it's important to know that, whenever the \verb!equals! method is overriden, we must provide a \verb!hashCode! method with the same header as shown in the code segment above. Java's enforcement of this ``policy" is known as a \vocab{hash code contract}. 


\subsection{Annotations}
In Java, \vocab{annotations} are used to help find runtime errors at compile time, and also provide additional information to others as to what you are trying to do. The general syntax for an annotation is \verb!@[annotation name]!. Some of the common ones used by the compiler are listed below:
\begin{itemize}
    \item \verb!@Test! indicates that a test follows.
    \item \verb!@Override! indicates that an overridden function follows.
    \item \verb!@SuppressWarnings! indicates that the compiler should suppress warnings surrounding the code that follows.
\end{itemize}

What's an example in which annotations can help us?

Suppose we're overriding a function. If we make a mistake and write the function header for the overridden wrong, our compiler won't interpret this as an error (assuming everything else is right). But, if we were to put the \verb!@Override! annotation before we override the function, our compiler will notice that we've got the wrong function header, and it will give us an error. 

This is particularly helpful because it helps us identify otherwise latent bugs.