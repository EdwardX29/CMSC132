\section{Friday, November 1, 2019}

\subsection{Binary Search Tree Implementation}
    
Recall that the \vocab{size} of a tree is the total number of nodes in the tree.  We can define a recursive function to determine the size of a tree as follows:


\begin{lstlisting}
	public int size() {
		return sizeAux(root);
	}

	public int sizeAux(Node rootAux) {
		return rootAux == null ? 0 : 1 + sizeAux(rootAux.left) + sizeAux(rootAux.right);
	}
\end{lstlisting}

This recursive function adds $1$ if the current node we're at isn't null, and it recurses on the left and right subtrees. Each node contributes $1$ to the size, which is exactly what we want.


Finally, here's a recursive function that computes the height of a tree (the maximum-lengthed path from the root to a leaf node):

\begin{lstlisting}
    /* Note: The height of an empty tree is not defined. */
	/* Source: https://xlinux.nist.gov/dads/HTML/height.html */
	public int height() {
		if (root != null) {
			return heightAux(root);
		}
		return -1;
	}
	
	public int heightAux(Node rootAux) {
		if (rootAux.left == null && rootAux.right == null) {
			return 0;
		} else if (rootAux.left != null && rootAux.right == null) {
			return 1 + heightAux(rootAux.left);
		} else if (rootAux.left == null && rootAux.right != null) {
			return 1 + heightAux(rootAux.right);
		} else {
			return 1 + Math.max(heightAux(rootAux.left), heightAux(rootAux.right));
		}
	}
\end{lstlisting}